\chapter{Implementation}\label{chapter:Implementation}
\section{Instance}\label{section:Data}
The data used for the parameters in the model implementation is the same as in \cite{mainreference_thies}, which represents an European passenger car manufacturer's setting, created with publicly available data and assumptions. Tables \ref{tab:vehicleprojectsparams}, \ref{tab:resourcesparams}, and \ref{tab:manufacturerparams} present some input values of the baseline model.
\begin{table}[h]
\centering
\begin{tabular}{l|l}
\toprule
Parameter&Input\\
\midrule
Vehicle Size&Small, Medium or Large \\
Powertrain Technology&BEV, FCEV, ICEV or PHEV\\
Power Class&Low or High\\
Fixed Development Cost&\euro{420} million\\
Period Length for Development Cost Distribution&3 years\\
Maximum Market Life Cycle&7 years\\
\bottomrule
\end{tabular}
\caption{Input values for Vehicle Projects in the baseline scenario}\label{tab:vehicleprojectsparams}
\end{table}
\begin{table}[h]
\centering
\begin{tabular}{l|l}
\toprule
Parameter&Input\\
\midrule
Fixed Ramp-up Cost&\euro{20} million \\
Variable Ramp-up Cost per Capacity Unit Increased&\euro{2,750}\\
Fixed Installed Capacity Cost per Unit per Period&\euro{50}\\
Fixed Development Cost&\euro{420} million\\
Percentage Available to Utilization in the Ramp-up Year&75\%\\
Period Length for Ramp-up Costs Distribution&5 years\\
Resource Depreciation Period&5 years\\
\bottomrule
\end{tabular}
\caption{Input values for Production Resources in the baseline scenario}\label{tab:resourcesparams}
\end{table}
\begin{table}[h]
\centering
\begin{tabular}{l|l}
\toprule
Parameter&Input\\
\midrule
Yearly Minimum Production Target&1 million vehicles\\
Maximum Realized Projects per Year&6 projects\\
Target Interest Rate&5\% per year\\
Tolerated Emission Excedeence&100\%\\
Emission Exceedance Penalty Cost& \euro{95} per gram and vehicle sold\\
\bottomrule
\end{tabular}
\caption{Input values for Manufacturer Settings and Emission Regulation in the baseline scenario}\label{tab:manufacturerparams}
\end{table}\\
\section{Main Assumptions}\label{sec:Main Assumptions}
The assumptions encompass information on vehicle projects, production resources, $\text{CO}_{2}$ fleet emission legislation, and the manufacturer's specific setting.\\

\subsection{Vehicle Projects}\label{vehicleprojects}
For each project, variable production cost and $\text{CO}_{2}$ emission rate based on powertrain technology, vehicle size, and power class are defined. On top of that, the variable production costs of  \gls{ICEV}s and \gls{PHEV}s are assumed to be increasing over time, and their $\text{CO}_{2}$ emission rates are decreasing over time because the internal combustion engine technology is regarded to be improved consistently. Owing to these technological improvements, additional investments are considered, leading to increasing variable production costs. Another trend being considered is for the \gls{BEV}s and \gls{FCEV}s, their variable production costs are decreasing over time because of the assumption of the further advancements in both battery and fuel cell technology (\cite{berckmans2017}, \cite{Lutsey2029}). On the other hand, the sales price for each project is assumed to be dependent only on the vehicle size and the power class.\\

\subsection{Demand}\label{demand}
The projects' demand is presumed as mutually exclusive market segments. The market segments are defined by three vehicle characteristics: size (small, medium, and large) with time-invariant market shares, power class (low and high) with equally distributed demand, and powertrain technology (\gls{ICEV}, \gls{PHEV}, \gls{BEV}, and \gls{FCEV}).\\
Six market scenarios are introduced in \cite{mainreference_thies}'s work. However, only one scenario is implemented in the model which assumes a fast increase in demand for \gls{ZEV}, reaching 63\% of the market share in the year 2035. Also, in this scenario, the market share of \gls{ZEV} is considered a preference for \gls{BEV} with a ratio of 75\% while 25\% for \gls{FCEV}. In addition, the market interest for \gls{PHEV}s grows at the same rate of demand decrease for \gls{ICEV}s.\\
\subsection{Production Resources}\label{productionresources}
For the automotive industry, the resource capacity is usually measured in units produced (\cite{mainreference_thies}). The resources are shared between projects with the same vehicle size, powertrain technology, and year of development; however, all resources are considered to have the same fixed and variable costs for simplification. For the calculation of cash flows' \gls{NPV}, the total ramp-up cost is considered to be distributed over the years before the first ramp-up year, and the resource value is linearly depreciated over the following years.\\
\subsection{Emission Legislation}\label{emissionlegislation}
The $\text{CO}_{2}$ fleet emission regulation parameters are based on the EU legislation published in 2019 for the baseline scenario. The emission thresholds are assumed to decrease through the years as stated in the Regulation EU 2019/631 (\cite{european2019regulation}). In this EU Regulation, the penalty payment is also set constant per g$\text{CO}_2$ exceedance and vehicle sold per year.\\
\subsection{Manufacturer's setting}\label{manufacturerssetting}
The assumptions for the manufacturer setting are the same as considered in \cite{mainreference_thies}. The planning horizon is from 2026 to 2035. The previously determined vehicle projects and installed capacity since 2020 are demonstrated in the Table \ref{tab:projectsbefore2026} and \ref{tab:resourcesbefore2026}.\\
\section{Computational Information}\label{sec: computational information}
The results presented in the following sections were generated by implementing the \gls{MIP} model in Gurobi 11.0, and the program has approximately 19,000 integer variables (thereof roughly 3,500 binary), about 200 continuous variables, and approximately 15,000 constraints for the instances analyzed. With a standard PC (AMD Ryzen 7 2700U, 2.2 GHz, 8 GB RAM, 64-Bit Windows 10), the model needs less than 0.5 minutes to be solved to optimality, which is fast enough given the strategic character of the \gls{PPP} task.\\
\section{Sensitivity Analysis of Development Cost}\label{section:Experiment 1}
One of the assumptions of the baseline model proposed by \cite{mainreference_thies} mentioned in section \ref{vehicleprojects} is that vehicles of every powertrain technology have the same development cost, \euro{420} million, but this doesn't align with practical evidence. Research indicates that \gls{BEV} and \gls{FCEV} generally incur higher development costs due to advancements in battery and fuel cell technologies (\cite{konig2021overview, wolfram2016electric}). Hence, it would be intriguing to explore the sensitivity of development cost stemmed from different powertrain technologies.
\subsection{$2^k$ Factorial Design}\label{Design 1}
The goal of the sensitivity analysis is to find out which factor among all four powertrain technology development costs has the largest effect on resulting \gls{NPV}. An approach proposed by \cite{law2007simulation}, so-called $2^k$ factorial design is to get an initial estimate how each factor affects the response and how each factor interacts with each other. However, using this approach requires a parameter study beforehand to decide on "+" and "-" level of each factor. Also, to avoid too much computational effort, a preliminary test to find out key factors for "+" and "-" scenarios can be helpful. To conduct the preliminary parameter study, the method of altering only the value of one factor -40\%, -20\%, +20\% and +40\% at a time, that is, ceteris paribus, is implemented.\\
After running the model with and without changes of development cost of each powertrain vehicle, the result is illustrated in the Figure \ref{fig:ParameterStudy}. Evident response of \gls{NPV} occurs when development cost changes from \euro{420} million to \euro{336} million (-20\%) or to \euro{503} million (+20\%). This shows the objective value is responsive enough. For the purpose of simplification, "+" level is decided to be +10\% and "-" level is -10\% for the $2^k$ factorial experiment. Furthermore, Figure \ref{fig:ParameterStudy} also implies that changing \gls{BEV} development cost or \gls{ICEV} development cost alone has greater impact on \gls{NPV} whereas changing \gls{PHEV} development cost yields little response. As a result, the factor: \gls{PHEV} development cost will not be considered in the $2^k$ factorial experiment. The experiment design is shown in the Table \ref{tab:2k design} with "+" representing increasing development cost of designated powertrain vehicle 10\% and "-" representing decreasing development cost 10\%.
\begin{figure}[h!]
\begin{center}
\includegraphics[width=0.9\textwidth]{images/Parameter Study.png}
\caption{Result of Parameter Study}
\label{fig:ParameterStudy}
\end{center}
\end{figure}

\begin{table}[h]
\centering
\begin{tabular}{cccc|c}
\toprule
Combination &$\Delta C_{RD, ICEV}$&$\Delta C_{RD, BEV}$&$\Delta C_{RD, FCEV}$& $\Delta NPV$(million €)\\
\midrule
1&-&-&-&809.86\\
2&+&-&-&184.54\\
3&-&+&-&103.59\\
4&-&-&+&575.72\\
5&+&+&-&-521.73\\
6&+&-&+&-49.6\\
7&-&+&+&-130.55\\
8&+&+&+&-755.87\\
\bottomrule
\end{tabular}
\caption{$2^k$ factorial design and \gls{NPV} response}\label{tab:2k design}
\end{table}

\subsection{Main Effect and Interaction Effect}\label{Result 1}
\begin{figure}[h!]
\begin{center}
\includegraphics[width=0.9\textwidth]{images/2k Factorial Result.png}
\caption{Result of $2^k$ factorial experiment}
\label{fig:FactorialResult}
\end{center}
\end{figure}
The graphical result of $2^k$ factorial experiment is illustrated in the Figure \ref{fig:FactorialResult} and the corresponding values are shown in the Table \ref{tab:2k design}. The first interpretation of the result is, for example, regarding to the Combination 1, that if the development costs of \gls{ICEV}, \gls{BEV} and \gls{FCEV} decrease \euro{42} million respectively, \gls{NPV} will increase \euro{809,86} million in total. To further analyze the result, the formulas of quantifying main effect and interaction effect are introduced in the Equation \ref{eq:main effect} and \ref{eq:interaction effect} where $\mathbb S=\{1, 2, ..., 8\}$ is the set of all eight combinations,  $\mathbb F=\{\Delta C_{RD,ICEV}, \Delta C_{RD,BEV}, \Delta C_{RD,FCEV}\}$ is the set of all the factors and $\Delta NPV_i$ is the \gls{NPV} difference to the baseline \gls{NPV} for Combination $i \in \mathbb S$ (\cite{law2007simulation}).
\begin{equation} \label{eq:main effect}
E_j = \frac{\sum_{i \in \mathbb S} f_{ij} * \Delta NPV_i}{2^{k-1}} \qquad \forall j \in \mathbb F
\end{equation}
\begin{equation} \label{eq:interaction effect}
E_{jh} = \frac{\sum_{i \in \mathbb S} f_{ij} *f_{ih} * \Delta NPV_i}{2^{k-1}} \qquad \forall j, h \in \mathbb F, j \not = h
\end{equation}
where
\begin{equation} \label{eq:fij}
f_{ij} = 
    \begin{cases}
          1 & \text{if factor $j$ is at the "+" level in the Combination $i$} \\
         -1 & \text{if factor $j$ is at the "-" level in the Combination $i$}\\
    \end{cases}\\ 
    \quad \forall j \in \mathbb F, \quad \forall i \in \mathbb S
\end{equation}
Calculation outcomes of quantified main effect and interaction effect of the three factors $\Delta C_{RD,ICEV}$, $\Delta C_{RD,BEV}$, $\Delta C_{RD,FCEV}$ are presented in the Table \ref{tab:main effect} and Table \ref{tab:interaction effect}. Values in the Table \ref{tab:main effect} can be interpreted as the change of NPV if the factor j changes from -10\% to +10\%, while the values in the Table \ref{tab:interaction effect} represent how much does one factor change if another changes from -10\% to +10\%.
\begin{table}[h]
\begin{minipage}{0.45\textwidth}
\centering
\begin{tabular}{c|c}
\toprule
$j$&$E_j$\\
\midrule
$\Delta C_{RD, ICEV}$&-625.32 \\
$\Delta C_{RD, BEV}$&-706.27 \\
$\Delta C_{RD, FCEV}$&-234.14 \\
\bottomrule
\end{tabular}
\caption{Main Effect}\label{tab:main effect}
\end{minipage}
\hfill
\begin{minipage}{0.45\textwidth}
\centering
\begin{tabular}{c|c}
\toprule
%\hline
$j$, $h$ & $E_{jh}$\\
\midrule
$\Delta C_{RD, ICEV}$, $\Delta C_{RD, BEV}$& 0\\
$\Delta C_{RD, BEV}$, $\Delta C_{RD, FCEV}$& 0\\
$\Delta C_{RD, FCEV}$, $\Delta C_{RD, ICEV}$& 0\\
\bottomrule
\end{tabular}
\caption{Interaction Effect}\label{tab:interaction effect}
\end{minipage}
\end{table}

\subsection{Managerial Insights}\label{Managerial Insights 1}
Based on the outcomes in the Table \ref{tab:main effect} and the Figure \ref{fig:ParameterStudy}, the sensitivity of \gls{BEV} development cost to \gls{NPV} is highest, followed by \gls{ICEV} development cost. In contrast, \gls{PHEV} development cost has a relatively minor effect on \gls{NPV}. Consequently, if a car manufacturer aims to reduce development costs, prioritizing \gls{BEV} development would be the most effective strategy, with \gls{PHEV} development considered as a last resort. For instance, reducing \gls{BEV} development costs from \euro{420} million to  \euro{336} million could lead to a substantial increase in \gls{NPV}, amounting to  \euro{737.96} million, as illustrated in Figure \ref{fig:ParameterStudy}. On the other hand, reducing the same amount of development costs for \gls{PHEV} could only result in \gls{NPV} increase of  \euro{119.49} million.  Furthermore, the Table \ref{tab:interaction effect} reveals that there is no significant interaction between \gls{ICEV}, \gls{BEV}, and \gls{FCEV} development cost changes. This is reasonable because alterations in the development cost of one type of powertrain do not appear to impact the development costs of other types.\\
\section{Sensitivity Analysis of Emission Penalty Cost}\label{section:Experiment 2}
One assumption of the baseline model proposed by \cite{mainreference_thies} mentioned in section \ref{emissionlegislation} is that the penalty cost for the exceedance of the $\text{CO}_{2}$ fleet emission threshold stays constant at \euro{95} per gram exceeded and vehicle sold through the entire planning horizon. However, the substantial accountability of the greenhouse gas emissions and the fossil-fueled transportation system to the climate change have been assessed, especially in the EU and in the USA (\cite{Urain2022}, \cite{Kinney2018179}, \cite{WENDEKER2022118439}). Thus, it is reasonable to imagine that the penalty cost could be increased in later versions of the EU Regulation (\cite{european2019regulation}) to increase the pressure for the European fleet decarbonization. On the contrary, the automotive industry is a major component of the total industrial value added to the European \glstext{GDP}, becoming vital for the European economy (\cite{Demiraj2022}). Considering this importance for the continent, the automotive industry could pressure the authorities to decrease the penalty cost to not hinder the thriving business in this economy section. In addition, Regulation (EU) 2019/631 (\cite{european2019regulation}) has been updated 8 times since its publication in April 2019, proving the possibility of constant changes in the regulations for current topics. In the baseline model the thresholds mentioned in section \ref{emissionlegislation}, which were calculated by \cite{mainreference_thies} is utilized. The penalty cost sensitivity experiment is conducted with the newly updated thresholds from Decision 2023/1623 (\cite{european2023newregulation}) to understand the impact of the new regulation standards and to compare the results with the two sets of threshold values, which are available in the Table \ref{tab:setsofthresholds}.\\

\subsection{Design}\label{designexperiment2}
With the two sets of threshold values - the ones from the 2019 Regulation and the ones from the 2023 Regulation, sensitivity experiments are implemented as shown in Table \ref{tab:scenariossecondexperiment}. First, Scenario 1 was run, in which the parameter $\varepsilon$ was equal to zero. It is possible to evaluate the model outcomes where it is ensured compliance with the $\text{CO}_{2}$ fleet emission thresholds. Consequently, the emission penalty cost is altered by -10\%, -5\%, +5\%, and + 10\% in each scenario to analyze the impacts in the model result. The main objective of this experiment is to assess the configuration of the production portfolio in terms of the percentage of each powertrain technology used in production per period. \\
% penalty cost in scenario 1 doesn't matter because constraint X is not being used

\begin{table}[h]
\centering
\begin{tabular}{c|ccc}
\toprule
Scenario &Penalty Cost Variation&Penalty Cost (\euro{} per g $\text{CO}_{2}$/km)&$\varepsilon$\\
\midrule
Baseline&0\% & 95,00&100\% \\
Scenario 1&--&--&0\%\\
Scenario 2&-10\%&85,50&100\%\\
Scenario 3&-5\%&90,25&100\%\\
Scenario 4&+5\%&99,75&100\%\\
Scenario 5&+10\%&104,50&100\%\\
\bottomrule
\end{tabular}
\caption{Experiment Design for Emission Penalty Cost Sensitivity Analysis}\label{tab:scenariossecondexperiment}
\end{table}

\subsection{Result}\label{resultexperiment2}
\subsubsection{Thresholds from 2019}\label{resultexperiment2old}
The \gls{NPV}s obtained for all 6 scenarios with the $\text{CO}_{2}$ fleet emission thresholds from 2019 can be seen in Figure \ref{fig:NPV for Old Thresholds}. It is more profitable to exceed the emission thresholds and; therefore, pay the penalty costs for it than to comply entirely with them. Respecting the thresholds would be as profitable as exceeding them if the penalty cost would be \euro{266} or higher. \\
\begin{figure}[H]
\begin{center}
\includegraphics[width=0.9\textwidth]{LaTeX_Vorlagen_Studienarbeiten/images/npv old thresholds.png}
\caption{\gls{NPV} under different scenarios using the thresholds from 2019}
\label{fig:NPV for Old Thresholds}
\end{center}
\end{figure}
When comparing the scenarios where it is allowed to exceed the thresholds between each other, it is possible to verify that there are no differences in the configuration of the production portfolio regarding the powertrain technology used. Furthermore, when enforcing total compliance with the thresholds, the difference in the powertrain chosen is small, due to only having penalty costs occured in periods 2026 and 2027. The only expressive difference in the production portfolio between scenarios where $\varepsilon$ equals 1 or 0 is the first year of the planning horizon: the share of \gls{ICEV}s needed decreases from 83,4\% to 79,8\% as seen in Figure \ref{fig:portfolio for old thresholds}.
\begin{figure}[h]
\begin{center}
\includegraphics[width=0.9\textwidth]{LaTeX_Vorlagen_Studienarbeiten/images/production portfolio old thresholds.png}
\caption{Production portfolio in terms of powertrain technology using the thresholds from 2019 when $\varepsilon$ = 1 (above) and $\varepsilon$ = 0 (below)}
\label{fig:portfolio for old thresholds}
\end{center}
\end{figure}

\subsubsection{Thresholds from 2023}\label{resultexperiment2new}
After running all six scenarios with the $\text{CO}_{2}$ fleet emission thresholds from 2023, \gls{NPV} under each scenario is illustrated in the Figure \ref{fig:NPV for New Thresholds}). Similar to the result presented in Section \ref{resultexperiment2old}, it is more profitable to exceed the emission thresholds; therefore, pay the penalty costs for it than to comply entirely with the regulation. Respecting the thresholds defined in 2023 would be as profitable as exceeding them if the penalty cost would be \euro{137} or higher. \\
\begin{figure}[h]
\begin{center}
\includegraphics[width=0.9\textwidth]{LaTeX_Vorlagen_Studienarbeiten/images/npv new thresholds.png}
\caption{\gls{NPV} under different scenarios using the thresholds from 2023}
\label{fig:NPV for New Thresholds}
\end{center}
\end{figure}
Similar to the outcome presented in Section \ref{resultexperiment2old},  there are no differences in the configuration of the production portfolio regarding the powertrain technology used when comparing the scenarios where it is allowed to exceed the thresholds between each other. In addition, when enforcing total compliance with the thresholds, the difference in the powertrain chosen is small, due to only having penalty costs in periods 2026, 2029, and 2030. The evident differences in the production portfolio between scenarios where $\varepsilon$ equals 1 or 0 are the share of \gls{ICEV}s needed to decrease from 57,5\% to 49,4\% in the period 2030, when this difference was partially covered with \gls{BEV}s and partially covered with \gls{FCEV}s, and from 53,2\% to 52,9\% in the period 2031, when it was overridden by \gls{BEV}s only, as seen in Figure \ref{fig:portfolio for new thresholds}.
%% the changed in the 2026 and 2029 were::
\begin{figure}[H]
\begin{center}
\includegraphics[width=0.9\textwidth]{LaTeX_Vorlagen_Studienarbeiten/images/production portfolio new thresholds.png}
\caption{Production portfolio in terms of powertrain technology using the thresholds from 2023 when $\varepsilon$ = 1 (above) and $\varepsilon$ = 0 (below)}
\label{fig:portfolio for new thresholds}
\end{center}
\end{figure}

\subsubsection{Comparison and Analysis}\label{resultexperiment2comparison}
After completing the two experiments, it is possible to compare both results and analyze the impact of the change in the regulation. With the \gls{NPV} plots (Figure \ref{fig:NPV for Old Thresholds} and Figure \ref{fig:NPV for New Thresholds}), one can identify that with the threshold from 2023, there is a lower incentive to exceed the thresholds, because the difference between the \gls{NPV}s from the scenario that it is possible to exceed and from the scenario that is it not is lower. Also, comparing the \gls{NPV} from the baseline scenario, the threshold from 2019 led to a more profitable outcome due to a higher percentage of \gls{ICEV}s on average, which are the powertrain technology of the most profitable vehicles available in the model.\\
With the new regulation, the year with the highest absolute variation in the threshold value is the last one in the planning horizon, the period 2035. With a threshold of zero in this period, the portfolio with the set of thresholds from 2023 ended with an entire portfolio with \gls{ZEV}, even in the scenario where $\varepsilon$ = 1. This happened because of the possibility of investing in resources for the \gls{ZEV}s through the planning horizon gradually.\\
Comparing the baseline scenarios for the two sets of thresholds, it is possible to identify that in both cases only in a few periods there was penalty cost payments, but in each scenario the periods that this happened were different. In the 2019 thresholds scenario, the emissions threshold exceedance only happened in 2026 and 2027, mainly due to the chosen projects and already installed resource capacities before the planning horizon started, which was entirely composed of \gls{ICEV}s. On the other hand, in the 2023 thresholds scenario, the periods with penalty costs were 2026, 2029, and 2030. The first one with the same cause as in the previous scenario and the others are because of the decrease in the threshold between 2029 and 2030 of approximately 47\%.\\

\subsection{Managerial Insights}\label{managerialinsightsexperiment2}
Given the decreasing thresholds through the periods, it is clear that the near future of mobility is \gls{ZEV}s, at least in Europe. Also, the companies that prepare themselves for it sooner with early investment in resources to produce these vehicles and with early development of vehicle models with this technology will achieve higher financial results. This is observed in the scenario with the threshold set of 2023 and $\varepsilon$ = 1 when there is 100\% of \gls{ZEV}s in 2035, because there has been enough time to prepare the manufacturing resources for this new era of the automotive industry with investments in \gls{BEV}s and \gls{FCEV}s.\\
Taken that exceeding at least one emission threshold was more financially beneficial with both sets of thresholds, the only reason to comply 100\% with the regulation thresholds with this problem setting would be the manufacturer strategy. If having a positive environmental impact was a goal for the manufacturer, then it would be worthwhile to undertake the small portfolio changes mentioned in the previous sections. Moreover, if having a diversified portfolio is also desired by the manufacturer, it could be convenient to add a new constraint to the model that ensures the election of at least one vehicle from each powertrain technology, since with the 2023 thresholds the optimal portfolio does not contain any \gls{PHEV}s.\\
After conducting this experiment, the importance of international regulations is clear since a lack of obligation to decrease the fleet emissions, would not incentivize the manufacturers to produce \gls{ZEV}s, or at least less polluting vehicles. This is true, especially because of the high profitability of \gls{ICEV}s and already installed production resources. Also, it is valuable to always be attentive to what is being discussed in the political sphere as the regulations can change. As soon as the manufacturer prepares its portfolio and resources for the newest limitations, the negative financial of the change impact may be smaller.