\chapter{Conclusion}\label{chapter:Conclusion}
This article implements the \gls{PPP} \gls{MIP} model from \cite{mainreference_thies}, which is a \gls{NPV} maximization problem that considers the cash flows from sales, production, development, and fleet emissions regulation. Also, this paper presents two experimental analyses, one focusing on the vehicle development cost and the other on the fleet emission penalty cost.\\
The sensitivity analysis regarding powertrain-specific development costs reveals that \gls{BEV}s exhibit the highest sensitivity to \gls{NPV}, followed by\gls{ICEV}s, \gls{FCEV}s, and \gls{PHEV}s. The second sensitivity analysis regarding the emission penalty cost uncovered the high profitability of \gls{ICEV}s. Even when the penalty cost increased by 10\%, the production of \gls{ICEV}s is still preferred in some periods. The second experiment also highlighted the importance of regulations that force companies to consider sustainable options in their portfolio. Without these regulations, industries may lack financial incentives to change how they do business. \\
There are many simplifications in this study. Demand is considered to have a time-invariant part, regarding vehicle sizes and power class, and a time-variant part, regarding the power train technology. However, demand forecasting is more complex than this, and requires a more in-depth analysis to be a better representation of the reality. Also, there is no minimum percentage of the demand that is supposed to be fulfilled in this implementation. Nevertheless, many manufacturers have the goal to remain an important player in particular market segments, especially the significant market share holders. Another simplification is the possibility of ramping up the resource capacities at any time without considering a limited budget or facility restrictions.\\
Future research articles could take into account these complexities. Forecasting vehicle demand already considering the most recent years of activities, between 2020 and 2023, would be a more realistic input data set for the model. Also, one could add constraints to ensure a minimal demand percentage for each market segment or to guarantee a diverse portfolio as mentioned in Section \ref{managerialinsightsexperiment2}. Last but not least, regarding to the production resources capacities, it is possible to add a constraint that limits the ramp-up size for each period to account for the limited available financial resources and finite shop floor.\\

